\chapter{Statistical Models}  
\label{chap:mixed} 

SHOW THE CODE FOR USING IT IN OUR 3 METHODS

GIVE A SIMPLE MATH DERIVATION OF VALUE OF IT, E.G. FOR SMALL N-I

SHOW THAT lm() GIVES THE SAME ANSWER AS MM

PROS AND CONS

Recommender systems is inherently statistical.  Indeed, the very fact
that we discuss the bias-variance tradeoff recognizes the fact that our
data are subject to sampling variation, a core statistical notion.  In
this chapter, we will apply classical statistical estimation methods to
a certain \textit{latent variables} model.

\section{The Basic Model}

Let $(U,I)$ denote a random (user ID, movie ID) pair.  Denote the user's
rating by $Y_{IJ}$.  The model is additive, postulating that

\begin{equation}
Y_{IJ} = \mu + \alpha_I + \beta_J + \epsilon
\end{equation}

Here $\mu$ is an unknown constant, the overall population mean over all
users and all movies.  The numbers $\alpha_1, \alpha_2,...$ and
$\beta_1, \beta_2,...$ are also unknown constants; think of $\alpha_i$
to be the tendency of user $i$ to give harsher ($\alpha_i < 0$) or more
generous ($\alpha_i > 0$) ratings, relative to the general population of
users, with a similar situation for the $\beta_j$ and movies.  The
$\epsilon$ term is thought of as the combination of all other affects.

Note that what makes, e.g., $\alpha_I$ random above is that $I$ is
random.

