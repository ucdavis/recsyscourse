\chapter{Linear Models}  
\label{chap:linmod} 

In Section \ref{rholargesmall} we made reference to \textit{linear
models}.  Directly or indirectly, they form the basis for much of ML.

\section{Minimizing MSPE}

Suppose we are predicting a random variable $Y$, based on a random
vector $X$, say predicting weight from (height,age).  Our guess will be
some function of $X$, $g(X)$.  What is the best $g$, in the sense of
minimizing MSPE?  In other words, how should we choose $g$ to minimize

\begin{equation}
\label{mspe}
E \left ( [Y - g(X)]^2 \right )
\end{equation}

% Using the Law of Iterated Expectations (Section \ref{adamsrule}), the
% above expression is equal to
% 
% \begin{equation}
% E \left [ E \left ( [Y - g(X)]^2 | X \right ) \right ]
% \end{equation}
% 
% For any random variable $W$, $E[ (W - c)^2$ is minimized by $c =
% EW$.\footnote{Write out the quantity as $E(W^2) -2c EW + (EW)^2$, then
% minimize with respect to $c$.}  Applying that to
% $E \left ( [Y - g(X)]^2 | X \right ) $, we have that:

One can show that:

\begin{quote}
We get minimum MSPE by taking $g(X)$ to be the conditional mean, $E(Y|X)$,
\end{quote}

This should make good intuitive sense:  To predict the weight of
someone 70 inches tall, we take our guess to be the mean weight of all
70-inch-tall people.

We will then define the \textit{regression function}

\begin{equation}
\mu(t) = E (Y | X = t)
\end{equation}

Note that the argument $t = (t_1,...,t_p)'$ is a vector.  
Also, prepending a 1, we will use the notation $\widetilde{t} = (1,t')'$.

This definition is \textit{general}; it does not assume a linear model.
Let's now see where the classical linear model comes from.

\section{Motivation for the Classical Linear Model}

Say we are predicting a variable $Y$ from $p$ features, $X_1,...,X_p$.
Let $X$ denote the (column) vector of those features.  Sometimes we will
also add a constant 1 at the beginning, setting

\begin{equation}
\widetilde{X} = (1,X_1,...,X_p)' 
\end{equation}

Here is the main point:

\begin{quote}
Suppose $(X,Y)$ has a $p+1$ dimensional normal distribution.  Then
the conditional distribution of $Y$ given $X$ has the following
properties:

\begin{itemize}

\item \textbf{Linearity:} 

\begin{equation}
\label{betadef}
\mu(t) = \beta_0 + \beta_1 t_1 + ... + \beta_p t_p  = \beta' \widetilde{t}
\end{equation}

for a certain vector ${\beta}$.\footnote{One can show that
$\beta = E(Y \widetilde{X}) [ E( \widetilde{X} \widetilde{X}') ]^{-1}$.}

\item \textbf{Conditional normality:}  For any $t$, the conditional
distribution of $Y$ given $X = t$ is normal.

\item \textbf{Homoskedasticity:}  The conditional
variance of $Y$ given $X = t$ is the same for all $t$.

\end{itemize} 

\end{quote}

\section{Coming Back Down to Earth}

For those readers who have some background in linear regression modeling, 
the above three bullet items will sound familiar.  They are the
motivation for the classic assumptions of linear modeling:  linearity,
normality and homoskedasticity.  \textit{However, note the following:}

\begin{itemize}

\item The normality and homoskedasticity assumptions are used only for
\textit{statistical inference}, i.e.\ confidence intervals and tests.
We will not be performing statistical inference in this
book; ML is mainly about \textit{prediction}.\footnote{Sadly, in ML
circles, the situation is confused by using the term \textit{inference}
for prediction.}  

Even for inference, those assumptions aren't really
necessary.  We get normality for our estimated $\beta$ from
the Central Limit Theorem, and we can deal with the lack of
homoskedasticity by using the \textit{sandwich estimator}, e.g.\ in the
R \textbf{sandwich} package. 

\item Many regresssion functions in practice are approximately linear.
And, as will be seen shortly, polynomial models, which may provide a
good fit in nonlinear situations, are actually linear!

\end{itemize} 

In other words:

\begin{quote}

\begin{itemize}

\item The classic linear model is motivated by settings in $(X,Y)$
has a multivariate normal distribution.

\item This assumption is highly restrictive, but \textit{is not
necessary} for our purposes.

\end{itemize} 

\end{quote}

For the remainder of this chapter, we assume 

\begin{equation}
\label{linassump}
g(t) = \beta' \widetilde{t}
\end{equation}

for some vector $\beta$ to be estimated from our data.

\section{Estimating $\beta$}

Here we lay groundwork leading up to the famous \textit{least-squares
estimate}.

\section{Sample vs.\ Population}

Most readers have probably noticed that when the results of a survey are
released, say during elections, a \textit{margin of error} (MOE) is stated.
For instance, ``55\% of those surveyed say they plan to vote for
Candidate Jones, with a margin of error of 3.2\%.''  The MOE is
recognition of the fact that only a sample of voters were surveyed, not
the entire population of voters.

Let $p$ denote the population proportion, i.e.\ the proportion of voters
across the population who favor Jones.  The value of $p$ is unknown, but
our estimate is $\widehat{p} = 0.55$.  The ``hat'' notation $\widehat{}$
means ``estimate of.''  (The MOE is the radius of a 95\% confidence
interval for $p$, though as noted, we do not make much use of statistical
inference in this book.)

Every ML method is an estimator in some form or other.  However, the
terms \textit{sample} and \textit{population} are not used in the ML
community.  Instead, they speak a \textit{probabilistic generative
process} to mean the same thing as sampling data from a population.

So, $\beta$ is a population value.  Its estimator from the data is
denoted $\widehat{\beta}$.

\subsection{The Least Squares Estimator}

Denote our data by $(X_{ij},Y_j), ~ j = 1,...,n$.  In other words, we
have $n$ data points, and in the $j^{th}$ of them, $X_{ij}$ and $Y_j$
are the valus of $X_i$ and $Y$, respectively.  Also, define

\begin{equation}
X^{(j)} = (1,X_{1j},...,X_{pj})'
\end{equation}

To keep a concrete example in mind, again suppose we are predicting
human weight $Y$ from height $X_1$ and age $X_2$.  $X^{(3)}$, for
instance, is then the vector (height,age) for the third person in our
data (with a 1 prepended).

Putting together the fact that $g$ minimizes (\ref{mspe}) and the
assumption (\ref{linassump}), we have that $\beta$ is the vector $b$ 
that minimizes

\begin{equation}
\label{linmspe}
E \left [ (Y - b' \widetilde{X})^2 \right ]
\end{equation}

The sample analog of this quantity is

\begin{equation}
\label{sammspe}
\frac{1}{n}
\sum_{j=1}^n \left [ (Y_j - b' X^{(j)})^2 \right ]
\end{equation}

Since $b = \beta$ minimizes (\ref{linmspe}), it is natural by analogy to
take $\widehat{\beta}$ to be the value of $b$ that minimizes
(\ref{sammspe}).

Just a little bit more notation:

\begin{equation}
D = (Y_1,...,Y_n)'
\end{equation}

\begin{equation}
A = 
\left (
\begin{array}{r}
X^{(1)'} \\
... \\
X^{(n)'} \\
\end{array}
\right )
\end{equation}

Then (\ref{sammspe}) (without the $1/n$ factor) is

\begin{equation}
\label{quad}
(D - Ab)'(D - Ab)
\end{equation}

To minimize this, we must set the derivative of (\ref{quad}) to 0.  It
can be shown that for a vector $u$, 

\begin{equation}
\frac{d}{du} u'u = 2u
\end{equation}

Applying this and the Chain Rule to (\ref{quad}), we have

\begin{equation}
0 = A (D - Ab) 
\end{equation}

Solve for $b$:

\begin{equation}
\widehat{\beta} = (A'A)^{-1} AD
\end{equation}

This is the \textit{least squares estimator} of $\beta$.
\section{The Logistic Model}

\section{Computation}

We certainly do not want to do this compution by hand.  What are our
choices?

\subsection{lm()}

R's \textbf{lm()} (``linear model'') function does the computation for
us.  Here is an example using \textbf{mlb}, a dataset in the
\textbf{regtools} package, involving Major League Baseball players:

\begin{lstlisting}
> lmout <- lm(Weight ~ Height + Age,mlb)
> coef(lmout)
 (Intercept)       Height          Age 
-187.6381754    4.9235994    0.9115326 
> predict(lmout,c(73,25))
Error in eval(predvars, data, env) : 
  numeric 'envir' arg not of length one
> predict(lmout,data.frame(Height=73,Age=25))
       1 
194.g
\end{lstlisting}

The call says, ``Fit a linear model for predicting weight from height
and age, using the dataset \textbf{mlb}.''  The \textbf{coef()} function
extracts $\widehat{\beta}$ from the output object.  We then predict a
new case.

By the way, \textbf{coef()} and \textbf{predict()} are \textit{generic}
functions in R, meaning that their actions depend on the class of object
they are called on.  In this case, \textbf{lmout} is of class
\textbf{'lm'}, so \textbf{coef()} is \textit{dispatched} to a function
\textbf{coef.lm()} tailored to such objects; \textbf{predict()} is
dispatched to \textbf{predict.lm()}.  There are many generic functions
in R, notably \textbf{print()} and \textbf{plot()}.

\subsection{qeLin()}

\section{Polynomial Models}

