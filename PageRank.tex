\chapter{PageRank}  
\label{chap:page}

Many attribute the early success of Google to its superior search
engine, based on an algorithm they called PageRank.  It was assigned US
Patent US6285999B1, with the sole inventor listed as Larry Page, one of
the two founders of the firm.  However, Page credited co-founder Sergei
Brin as contributing in conversations about the idea, as well as some
Stanford University faculty and some previous researchers who had done
related work.  The patent belongs to Stanford.  The name,
\textit{PageRank}, is a pun on both the inventor's surname and the fact
that it indexes Web \textit{pages}.

The goal is to rank all the pages of thw World Wide Web in order of
importance.  The key is defining that latter term.

\section{Markov Model}

The method models Web ``surfing'' as following a \textit{Markov chain}, a
very widely-used approach to stochastic modeling in many fields.
Its main assumption is rather strict, but in this case that is not very
important, as we are merely putting together a rough measure of the
importance of each Web page.

\subsection{Structure}

One observes some process at times 1,2,3,...  The \textit{state} at time
\textit{m}, a random variable, is denoted by $X_m$.  Different
applications have different state spaces; it could be queue length in a
network buffer, for instance.

The main assumption is that the system follows the \textit{Markov property},
which in rough terms can be described as:

\begin{quote}
The probabilities of future states, given the present state and the past
states, depends only on the present state; the past is irrelevant.
\end{quote}

We sometimes say the system is ``memoryless'' --- in looking to the
future, knowing the present, the past is ``forgotten.''

A famous example is \textit{random walk}.  Say at time 0 we are at
position 0 on the real number line.  We flip a coin, then go 1 step left
or right, depending on whether the outcome is head or tails.  Then we do
that again, repeatedly.  $X_m$ is our position at time $m$.  Clearly
this system is Markovian.

In formal terms:

\begin{equation}
\label{containshidden}
P(X_{t+1}=s_{t+1}|X_{t}=s_{t},X_{t-1}=s_{t-1},\ldots ,X_{0}=s_{0})=P(X_{t+1}=s_{t+1}|X_{t}=s_{t})
\end{equation}

\subsection{Matrix Formulation}

We define $p_{ij}$ to be the probability of going from state i to state
j in one time step; note that this is a {\it conditional} probability,
i.e. $P(X_{n+1} = j ~|~ X_n = i)$.  These quantities form a matrix,
denoted by $P$.

So, this matrix gives the probabilites of going from state $i$ to state
$j$ in one time step:

\begin{equation}
P(X_{n+1} = s ~|~ X_n = r) = P_{rs} = p_{rs}
\end{equation}

And the Markov property can be used to show that $P^2$ gives the
probabilities of the various 2-time step transitions, i.e.

\begin{equation}
P(X_{n+2} = s ~|~ X_n = r) = (P^2)_{rs}
\end{equation}

Similarly, $P^k$ gives the k-step probabilities,

\begin{equation}
\label{kstep}
P(X_{n+k} = s ~|~ X_n = r) = (P^k)_{rs}
\end{equation}

\subsection{Long-Run Behavior}

Suppose the chain is \textit{irreducible}, meaning that one can reach
every state from every other, in a finite number of
steps.\footnote{Computer scientists, true to form, have their own
separate term for this, \textit{strongly connected}.}  Suppose also that
the state space is finite, and \textit{aperiodic}.  The latter term
means that the GCD of return times to any state is 1; the random walk is
periodic, since that number is 2.  Then 

\begin{equation}
\label{limpi}
\pi_r = \lim_{m \rightarrow \infty} P(X_m = r)
\end{equation}

exists, and is independent of the starting position $X_0$.

Note that $\pi$ is a vector, with element $r$ defined as above.

\subsection{Finding $\pi$}

(\ref{limpi}) and (\ref{kstep}), and our by-now well-honed skill at
using partitioned matrices imply that 

\begin{equation}
\lim_{m \rightarrow \infty} P^m = 
\left (
\begin{array}{r}
\pi' \\
\pi' \\
... \\
\pi' \\
\end{array}
\right )
\end{equation}

Note that the fact the rows are identical here reflects the independence
of (\ref{limpi}) of the starting position $X_0$.

This gives us a way to find $\pi$:  Simply raise $P$ to a high power,
and then each row is approximately $\pi'$.  In fact, we could the get an
even better approximation by averaging those rows.  (Of course, this is
for chains with finite state spaces.)

Alternatively, one can view the whole thing as an eigenvalue problem.
The above material can be used to show that

\begin{equation}
\pi' = \pi P
\end{equation}

so that 

\begin{equation}
\pi' = P' \pi' 
\end{equation}

which makes $\pi'$ an eigenvector of $P'$ with eigenvalue 1.  It's also
possible to show that the other eigenvalues are between 0 and 1.  So an
alternate way to find $\pi$ is to find the eigenvectors of $P'$.  And in
turn one way to do that is to use the famous Power Method for finding
the eigenvector corresponding to the largest eigenvalue of a matrix.
That method involves powers of $P$ as well, so in the end it's similar
to the above.

\section{The PageRank Model}

Here the state space consists of one state for every page on the Web,
and ``time'' is hops from one page to another.  $X_3$, say, is the third
page you visit during a Web surfing session.  What Page had to do was
devise some clever formulation of the matrix $P$ that would reasonably
model Web surfing.

\subsection{Details}

The Markov property is questionable here, but again, we don't care much
about that.  It's just a mechanism for devising an importance measure.

Now, what is the matrix $P$ here?  For each Web page $i$, let $n_i$
denote the number of outlinks from that page.  The model is that after a
visit to that page, the Web surfer will choose one of those links, with
uniform probability among them, i.e.\ $1/n_i$ each.  The $\pi$ vector
then becomes the vector of importance scores for the various pages.

Google also adds a \textit{damping} factor to the model, which we will
not go into here.

\subsection{Computation of $\pi$}

All of the above is fine, but there are literally billions of Web pages.
Thus the matrix $P$ has billions of rows and columns.  Fortunately,
though, it is a very sparse matrix, lots of 0s, so we can exploit that
property to reduce computation.  

Note too that there is a shortcut to find the matrix powers:  We first
square $P$, then square the result, yielding $P^4$, then square
\textit{that} result, yielding $P^8$ and so on.

There are many refinements of all this, of course.

\section{Then What?}

Clearly there is a lot more to what Google does to provide users with
recommended Web page.  What we saw here is only a first step.
